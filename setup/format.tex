\documentclass[12pt,a4paper,openany,twoside,below,section]{book}

% ======= 页面和版面设置 =======
\usepackage{geometry}   % 设置页边距
\geometry{paperwidth=210mm,paperheight=297mm,%
  left=2.5cm,right=2.5cm,top=2.54cm,bottom=2.54cm}
\topmargin=-10.4mm
\headheight=17pt
\footskip=8mm
\headsep=5mm

\usepackage[below]{placeins}  % 防止浮动体跨越章节
\usepackage{float}  % 图片浮动位置控制

% ======= 图形和表格 =======
\usepackage{graphicx}  % 插图
\graphicspath{{figures/}}  % 预定义图片路径
\usepackage{subfigure}  % 多子图支持
\usepackage{rotating}  % 旋转图形
\usepackage{tabularx}  % 表格扩展
\usepackage{multirow}  % 表格多行合并
\usepackage{setspace}  % 行距设置
\usepackage[ruled,linesnumbered]{algorithm2e}
\usepackage{booktabs}  % 绘制三线表
\setlength\heavyrulewidth{0.35ex}

% ======= 文字编码和字体 =======
\usepackage{fontspec,xltxtra,xunicode}
\defaultfontfeatures{Mapping=tex-text} % 支持连字符等特殊符号
\usepackage{xeCJK}  % 中文支持
\usepackage{ctex}

% 中文字体配置
\setmainfont{Times New Roman}
\setsansfont{Times New Roman}
\setmonofont{Courier New}
\setCJKmainfont[AutoFakeBold=true]{SimSun}  % 宋体
\setCJKsansfont{SimHei}
\setCJKmonofont{FangSong}

\setCJKfamilyfont{hei}{SimHei}               % 黑体
\setCJKfamilyfont{kai}{KaiTi}                % 楷体
\setCJKfamilyfont{song}[AutoFakeBold]{SimSun}  % 宋体
\setCJKfamilyfont{fang}{FangSong}            % 仿宋体
\setCJKfamilyfont{enroman}{Times New Roman}

\newcommand{\hei}{\CJKfamily{hei}}
\newcommand{\kai}{\CJKfamily{kai}}
\newcommand{\song}{\CJKfamily{song}}
\newcommand{\fang}{\CJKfamily{fang}}
\newcommand*{\mytimes}{\CJKfamily{enroman}}

\newfontfamily\codefont{Courier New}
\newfontfamily\pagella{Times New Roman}

% ======= 字号、行距定义 =======
\newcommand{\yihao}{\fontsize{28pt}{36pt}\selectfont}     % 一号
\newcommand{\erhao}{\fontsize{21pt}{28pt}\selectfont}     % 二号
\newcommand{\sanhao}{\fontsize{16pt}{20pt}\selectfont}    % 三号
\newcommand{\xiaosan}{\fontsize{15pt}{22pt}\selectfont}   % 小三
\newcommand{\sihao}{\fontsize{14pt}{20pt}\selectfont}     % 四号
\newcommand{\xiaosi}{\fontsize{12pt}{20pt}\selectfont}    % 小四
\newcommand{\wuhao}{\fontsize{10.5pt}{10.5pt}\selectfont} % 五号
\newcommand{\xiaowu}{\fontsize{9pt}{9pt}\selectfont}      % 小五

\renewcommand{\baselinestretch}{1.0}  % 将行距缩放系数设为1,避免中文默认1.3导致的行距放大,使设定的20pt行距实际生效(否则为26pt),需配合 \selectfont 使用

% ======= 颜色和超链接 =======
\usepackage{xcolor}
\usepackage{color}  % 也可删除,xcolor已包含
\usepackage[unicode]{hyperref}
\hypersetup{
  hidelinks,
  CJKbookmarks=true,
  bookmarksnumbered=true,
  bookmarksopen=true,
  colorlinks=true,
  pdfborder=001,
  citecolor=blue,
  linkcolor=blue,
  anchorcolor=green,
  urlcolor=blue,
  pdfcreator={XeTeX,XeCJK}
}

\usepackage{caption}
%按NWPU标准, 图表标题字号为五号
\DeclareCaptionFont{tiZhuZiTi}{\fontsize{10.5}{10.5}\mdseries}
% 按NWPU标准, 设置题注分隔符
\DeclareCaptionLabelSeparator{tiZhuFenGe}{\hspace{0.5em}}
%重新定义caption的属性
\captionsetup{font=tiZhuZiTi, labelsep=tiZhuFenGe, skip=10pt}

% ======= 参考文献 =======
\usepackage[numbers,super,square,comma,sort&compress]{natbib}

% ======= 代码、下划线等特殊格式 =======
\usepackage{listings}  % 代码插入
\usepackage{ulem}      % 下划线、波浪线等

% ======= 画图 =======
\usepackage{tikz}

% ======= 其他 =======
\usepackage{wallpaper}  % 背景图片
\addtolength{\wpYoffset}{8.0cm}
\usepackage{url}
\usepackage{epsfig}     % eps图像支持
\usepackage{flafter}    % 浮动体显示顺序
\usepackage{calc}       % 计算
\usepackage{units}      % 单位宏包

% ======= 数学符号 =======
\usepackage{amsmath,amssymb}
\usepackage{mathtools}
\usepackage{bm}  % 粗斜体

% ======= 行距与段落设置 =======
% \linespread{1.4}
% \setlength{\parskip}{0.5\baselineskip}
\sloppy

% 段落首行缩进2个中文字符
\makeatletter
\let\@afterindentfalse\@afterindenttrue
\@afterindenttrue
\makeatother
\setlength{\parindent}{2em}

% ======= 页眉页脚 =======
\usepackage{fancyhdr}
\pagestyle{fancy}
\fancyhf{}  % 清空默认
\fancyfoot[C]{\thepage}  % 页脚居中显示页码
\renewcommand{\headrulewidth}{0pt}  % 去掉页眉线
\renewcommand{\footrulewidth}{0pt}  % 去掉页脚线

% ======= 图表编号按章节 =======
\usepackage{chngcntr}
\counterwithin{figure}{chapter}
\counterwithin{table}{chapter}
\counterwithin{equation}{chapter}
\renewcommand{\thefigure}{\thechapter-\arabic{figure}}
\renewcommand{\thetable}{\thechapter-\arabic{table}}
\renewcommand{\theequation}{\thechapter-\arabic{equation}}

% ======= 章节标题格式 =======
\setcounter{secnumdepth}{4}
\input{setup/gb_452.cpx}
\renewcommand\chaptername{\CJKprechaptername\CJKthechapter\CJKchaptername}



\usepackage{titlesec}

\titleformat{\chapter}[hang]
    {\filcenter \hei \fontsize{16pt}{20pt}\selectfont}
    {\hei \fontsize{16pt}{20pt}\selectfont{\chaptertitlename}}
    {16pt}{}
\titlespacing{\chapter}{0pt}{-2pt}{14pt}


\titleformat{\section}[hang]{\hei \sihao}
  {\sihao \thesection}{0.5em}{}
\titlespacing{\section}{0pt}{10pt}{0em}

\titleformat{\subsection}[hang]{\hei \xiaosi}
  {\xiaosi \thesubsection}{0.5em}{}
\titlespacing{\subsection}{0pt}{10pt}{0em}

% ======= 目录格式修改 =======
\usepackage{etoolbox}
\usepackage{titletoc}

\setcounter{tocdepth}{2}
\renewcommand{\contentsname}{目~~~~录}
\dottedcontents{chapter}[0.0em]{}{0.0em}{5pt}
\dottedcontents{section}[1.16cm]{}{1.8em}{5pt}
\dottedcontents{subsection}[2.00cm]{}{2.7em}{5pt}

\raggedbottom
% 调整列表环境的垂直间距和水平缩进
% labelindent=26pt, 两个小四字符是2*12pt=24pt,再修正2pt以求避免编号数字有突出感。下同。
\usepackage{enumitem}
\setlist[enumerate]{itemsep=0pt, topsep=0pt, partopsep=0pt, parsep=0pt,
                    labelindent=26pt, leftmargin=*, align=left}
\setlist[itemize]{itemsep=2pt, topsep=2pt, partopsep=2pt, parsep=2pt}
                %  labelindent=26pt, leftmargin=*, align=left}
\setlist[description]{itemsep=0pt, topsep=0pt, partopsep=0pt, parsep=0pt,
                      labelindent=26pt, leftmargin=*, align=left}

% 设置中文风格的章节引用
\newcommand{\chref}[1]{\CJKnumber{\ref{#1}}}
% 调整中文括号的间距
\newcommand{\KH}[1]{\!\!(#1)\!\!}
% 重设表格的纵向尺寸
\renewcommand\arraystretch{1.25}

% ======= 注释掉的示例标题格式 =======
% \title{\yihao{这里是大标题啊}}
% \author{\xiaoerhao{作者姓名}\footnote{电子邮件: ***@***.com,学号: ***}\\[2ex]
% \sanhao 东南大学网络空间安全学院\\[2ex]
% }
% \date{\sanhao\today}

